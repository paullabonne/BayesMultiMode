\nonstopmode{}
\documentclass[a4paper]{book}
\usepackage[times,inconsolata,hyper]{Rd}
\usepackage{makeidx}
\usepackage[utf8]{inputenc} % @SET ENCODING@
% \usepackage{graphicx} % @USE GRAPHICX@
\makeindex{}
\begin{document}
\chapter*{}
\begin{center}
{\textbf{\huge Package `BayesMultiMode'}}
\par\bigskip{\large \today}
\end{center}
\inputencoding{utf8}
\ifthenelse{\boolean{Rd@use@hyper}}{\hypersetup{pdftitle = {BayesMultiMode: Bayesian Mode Inference}}}{}
\ifthenelse{\boolean{Rd@use@hyper}}{\hypersetup{pdfauthor = {Nalan Baştürk; Jamie Cross; Peter de Knijff; Lennart Hoogerheide; Paul Labonne; Herman van Dijk}}}{}
\begin{description}
\raggedright{}
\item[Type]\AsIs{Package}
\item[Title]\AsIs{Bayesian Mode Inference}
\item[Version]\AsIs{0.5.0}
\item[Description]\AsIs{A bayesian approach for mode inference which works in two steps. First, a mixture distribution is
fitted on the data using a sparse finite mixture (SFM) Markov chain Monte Carlo
(MCMC) algorithm following Malsiner-Walli, Frühwirth-Schnatter and Grün (2016)
<}\Rhref{https://doi.org/10.1007/s11222-014-9500-2}{doi:10.1007/s11222-014-9500-2}\AsIs{>). The number of mixture components does not have
to be specified; the size of the mixture is estimated endogenously through the SFM
approach. Second, the modes of the estimated mixture at each MCMC draw are retrieved
using algorithms specifically tailored for mode detection. These estimates are then
used to construct posterior probabilities for the number of modes, their locations
and uncertainties, providing a powerful tool for mode inference.}
\item[License]\AsIs{GPL (>= 3)}
\item[Imports]\AsIs{assertthat,
bayesplot,
dplyr,
ggplot2,
ggpubr,
gtools,
magrittr,
MCMCglmm,
mvtnorm,
posterior,
sn,
stringr,
tidyr,
Rdpack,
scales}
\item[Depends]\AsIs{R (>= 3.4.0)}
\item[Suggests]\AsIs{testthat (>= 3.0.0)}
\item[RdMacros]\AsIs{Rdpack}
\item[Encoding]\AsIs{UTF-8}
\item[LazyData]\AsIs{true}
\item[URL]\AsIs{}\url{https://github.com/paullabonne/BayesMultiMode}\AsIs{}
\item[BugReports]\AsIs{}\url{https://github.com/paullabonne/BayesMultiMode/issues}\AsIs{}
\item[NeedsCompilation]\AsIs{no}
\item[RoxygenNote]\AsIs{7.2.3}
\item[Config/testthat/edition]\AsIs{3}
\end{description}
\Rdcontents{\R{} topics documented:}
\inputencoding{utf8}
\HeaderA{bayes\_estimation}{Bayesian estimation of mixture distributions}{bayes.Rul.estimation}
%
\begin{Description}\relax
Gibbs samplers for sparse finite mixture Markov chain Monte Carlo (SFM MCMC) estimation.
\end{Description}
%
\begin{Usage}
\begin{verbatim}
bayes_estimation(
  data,
  K,
  dist,
  priors = list(),
  nb_iter = 2000,
  burnin = nb_iter/2,
  printing = TRUE
)
\end{verbatim}
\end{Usage}
%
\begin{Arguments}
\begin{ldescription}
\item[\code{data}] Vector of observations

\item[\code{K}] Maximum number of mixture components

\item[\code{dist}] String indicating the distribution of the mixture components
Currently supports "normal", "skew\_normal", "poisson" and "shifted\_poisson"

\item[\code{priors}] List of priors; default is an empty list which implies the following priors :\\{}
a0 = 1,\\{} A0 = 200,\\{} b0 = median(y),\\{} B0 = (max(y) - min(y))\textasciicircum{}2 (normal),\\{}
D\_xi = 1,\\{} D\_psi =1, (skew normal: B0 = diag(D\_xi,D\_psi)), \\{} c0 = 2.5,\\{}
l0 = 1.1 (poisson),\\{} l0 = 5 (shifted poisson),\\{} L0 = 1.1/median(y),\\{} L0 = l0 - 1 (shifted poisson),\\{}
g0 = 0.5,\\{} G0 = 100*g0/c0/B0 (normal),\\{} 
G0 = g0/(0.5*var(y)) (skew normal)

\item[\code{nb\_iter}] Number of MCMC iterations; default is 2000

\item[\code{burnin}] Number of MCMC iterations used as burnin; default is nb\_iter/2

\item[\code{printing}] Showing MCMC progression ?
\end{ldescription}
\end{Arguments}
%
\begin{Value}
A list of class `BayesMixture` containing
\begin{itemize}

\item{} data - Same as argument
\item{} dist\_type - Type of the distribution (continuous or discrete)
\item{} pars\_names - Names of the mixture components' parameters
\item{} mcmc - Matrix of MCMC draws where the rows corresponding to burnin have been discarded
\item{} mcmc\_all - Original matrix of MCMC draws

\end{itemize}

\end{Value}
%
\begin{References}\relax
Malsiner-Walli G, Fruhwirth-Schnatter S, Grun B (2016).
``Model-based clustering based on sparse finite Gaussian mixtures.''
\emph{Statistics and Computing}, \bold{26}(1), 303--324.
ISSN 1573-1375, \Rhref{https://doi.org/10.1007/s11222-014-9500-2}{doi:10.1007\slash{}s11222\-014\-9500\-2}.\\{}\\{}
Frühwirth-Schnatter S, Pyne S (2010).
``Bayesian inference for finite mixtures of univariate and multivariate skew-normal and skew-t distributions.''
\emph{Biostatistics}, \bold{11}(2), 317--336.
ISSN 1465-4644, \Rhref{https://doi.org/10.1093/biostatistics/kxp062}{doi:10.1093\slash{}biostatistics\slash{}kxp062}, \url{2023-02-22}.\\{}\\{}
Frühwirth-Schnatter S, Malsiner-Walli G (2019).
``From here to infinity: sparse finite versus Dirichlet process mixtures in model-based clustering.''
\emph{Advances in Data Analysis and Classification}, \bold{13}, 33--64.\\{}\\{}
Viallefont V, Richardson S, Peter
J (2002).
``Bayesian analysis of Poisson mixtures.''
\emph{Journal of Nonparametric Statistics}, \bold{14}(1-2), 181--202.
\end{References}
%
\begin{Examples}
\begin{ExampleCode}
# Example with galaxy data ================================================
set.seed(123) 

# retrieve galaxy data
y = galaxy

# estimation
bayesmix = bayes_estimation(data = y,
                           K = 5, #not many to run the example rapidly
                           dist = "normal",
                           nb_iter = 500, #not many to run the example rapidly
                           burnin = 100)
                           
# plot estimated mixture
# plot(bayesmix, max_size = 200)

# Example with DNA data =====================================================

set.seed(123) 

# retrieve DNA data
y = d4z4

# estimation
bayesmix = bayes_estimation(data = y,
                           K = 5, #not many to run the example rapidly
                           dist = "shifted_poisson",
                           nb_iter = 500, #not many to run the example rapidly
                           burnin = 100)
                           
# plot estimated mixture
# plot(bayesmix, max_size = 200)


\end{ExampleCode}
\end{Examples}
\inputencoding{utf8}
\HeaderA{bayes\_mode}{Bayesian mode inference}{bayes.Rul.mode}
%
\begin{Description}\relax
Estimates modes for each mcmc draws which are then used to compute posterior probabilities for the number of modes and their locations.
The fixed-point algorithm of Carreira-Perpinan (2000) is used for Gaussian mixtures
while the Modal EM algorithm of Li et al. (2007) is used for other continuous mixtures.
\end{Description}
%
\begin{Usage}
\begin{verbatim}
bayes_mode(
  BayesMix,
  rd = 1,
  tol_x = sd(BayesMix$data)/10,
  show_plot = FALSE,
  nb_iter = NULL
)
\end{verbatim}
\end{Usage}
%
\begin{Arguments}
\begin{ldescription}
\item[\code{BayesMix}] An object of class `BayesMixture`

\item[\code{rd}] Rounding parameter

\item[\code{tol\_x}] Tolerance parameter for distance in-between modes; default is sd(data)/10 where data is an element of BayesMix.
If two modes are closer than tol\_x, only the first estimated mode is kept.
Not needed for mixtures of discrete distributions.

\item[\code{show\_plot}] Show density with estimated mode as vertical bars ?

\item[\code{nb\_iter}] Number of draws on which the mode-finding algorithm is run; default is NULL which means the algorithm is run on all draws.
\end{ldescription}
\end{Arguments}
%
\begin{Value}
A list of class `BayesMode` containing
\begin{itemize}

\item{} data - from BayesMix argument
\item{} dist - from BayesMix argument
\item{} dist\_type - from BayesMix argument
\item{} pars\_names - from BayesMix argument
\item{} modes - Matrix with a row for each draw and columns showing modes
\item{} p1 - Posterior probability of unimodality
\item{} tb\_nb\_modes - Matrix showing posterior probabilities for the number of modes
\item{} table\_location - Matrix showing the posterior probilities for location points being modes

\end{itemize}

\end{Value}
%
\begin{References}\relax
Carreira-Perpinan MA (2000).
``Mode-finding for mixtures of Gaussian distributions.''
\emph{IEEE Transactions on Pattern Analysis and Machine Intelligence}, \bold{22}(11), 1318--1323.
ISSN 1939-3539, \Rhref{https://doi.org/10.1109/34.888716}{doi:10.1109\slash{}34.888716}, Conference Name: IEEE Transactions on Pattern Analysis and Machine Intelligence.\\{}\\{}
Li J, Ray S, Lindsay BG (2007).
``A Nonparametric Statistical Approach to Clustering via Mode Identification.''
\emph{Journal of Machine Learning Research}, \bold{8}, 1687-1723.
\end{References}
%
\begin{Examples}
\begin{ExampleCode}
# Example with galaxy data ================================================
set.seed(123) 

# retrieve galaxy data
y = galaxy

# estimation
bayesmix = bayes_estimation(data = y,
                           K = 5, #not many to run the example rapidly
                           dist = "normal",
                           nb_iter = 500, #not many to run the example rapidly
                           burnin = 100)

# mode estimation
bayesmode = bayes_mode(bayesmix)

# plot 
# plot(bayesmode, max_size = 200)

# summary 
# summary(bayesmode)

# Example with DNA data ================================================
set.seed(123) 

# retrieve galaxy data
y = d4z4

# estimation
bayesmix = bayes_estimation(data = y,
                           K = 5, #not many to run the example rapidly
                           dist = "shifted_poisson",
                           nb_iter = 500, #not many to run the example rapidly
                           burnin = 100)

# mode estimation
bayesmode = bayes_mode(bayesmix)

# plot 
# plot(bayesmode, max_size = 200)

# summary 
# summary(bayesmode)

\end{ExampleCode}
\end{Examples}
\inputencoding{utf8}
\HeaderA{bayes\_trace}{Trace plots}{bayes.Rul.trace}
%
\begin{Description}\relax
This is wrapper around the `mcmc\_trace()` function from package bayesplot.
\end{Description}
%
\begin{Usage}
\begin{verbatim}
bayes_trace(BayesMix, mcmc_vars = NULL, with_burnin = FALSE, ...)
\end{verbatim}
\end{Usage}
%
\begin{Arguments}
\begin{ldescription}
\item[\code{BayesMix}] An object of class BayesMixture

\item[\code{mcmc\_vars}] Variables to plot; default is all the variable in the MCMC output

\item[\code{with\_burnin}] Plot all draws ?

\item[\code{...}] Additional arguments passed to function `mcmc\_trace()` from the package bayesplot.
\end{ldescription}
\end{Arguments}
%
\begin{Value}
A trace plot.
\end{Value}
%
\begin{Examples}
\begin{ExampleCode}
# Example with galaxy data ================================================
set.seed(123) 

# retrieve galaxy data
y = galaxy

# estimation
bayesmix = bayes_estimation(data = y,
                           K = 5, #not many to run the example rapidly
                           dist = "normal",
                           nb_iter = 500, #not many to run the example rapidly
                           burnin = 100)

# trace plot
bayes_trace(bayesmix)

\end{ExampleCode}
\end{Examples}
\inputencoding{utf8}
\HeaderA{ct47}{X chromosomal macrosatellite repeats ct47}{ct47}
\keyword{datasets}{ct47}
%
\begin{Description}\relax
Repeat units that encode for a cancer testis antigen.\\{}
Locus (hg18): Xq24 \\{}
Unit (kb): 4.8 \\{}
Restriction enzyme: EcoRI \\{}
Encoded product : cancer testis antigen 47
\end{Description}
%
\begin{Usage}
\begin{verbatim}
ct47
\end{verbatim}
\end{Usage}
%
\begin{Format}
A vector of counts with 410 elements.
\end{Format}
%
\begin{References}\relax
Schaap M, Lemmers RJ, Maassen R, van der Vliet PJ, Hoogerheide LF, van Dijk HK, Basturk N, de Knijff P, van der Maarel SM (2013).
``Genome-wide analysis of macrosatellite repeat copy number variation in worldwide populations: evidence for differences and commonalities in size distributions and size restrictions.''
\emph{BMC Genomics}, \bold{14}(1), 143.
ISSN 1471-2164, \Rhref{https://doi.org/10.1186/1471-2164-14-143}{doi:10.1186\slash{}1471\-2164\-14\-143}.
\end{References}
\inputencoding{utf8}
\HeaderA{cyclone}{Tropical cyclones lifetime maximum intensity}{cyclone}
\keyword{datasets}{cyclone}
%
\begin{Description}\relax
Dataset constructed using the International Best Track Archive for Climate Stewardship (IBTrACS).
The distribution of tropical cyclones lifetime maximum intensity across the globe is known
to be bimodal which has important implications for climate modelling.
\end{Description}
%
\begin{Usage}
\begin{verbatim}
cyclone
\end{verbatim}
\end{Usage}
%
\begin{Format}
A dataset with three columns showing the identification of the cyclone, its year of occurrence and its lifetime maximum intensity (LMI).
LMI is calculated as the maximum wind speed for each cyclone with unit ks.
\end{Format}
%
\begin{Source}\relax
https://www.ncei.noaa.gov/products/international-best-track-archive
\end{Source}
%
\begin{References}\relax
Knapp KR, Kruk MC, Levinson DH, Diamond HJ, Neumann CJ (2010).
``The International Best Track Archive for Climate Stewardship (IBTrACS): Unifying Tropical Cyclone Data.''
\emph{Bulletin of the American Meteorological Society}, \bold{91}(3), 363--376.
ISSN 0003-0007, 1520-0477, \Rhref{https://doi.org/10.1175/2009BAMS2755.1}{doi:10.1175\slash{}2009BAMS2755.1}, Publisher: American Meteorological Society Section: Bulletin of the American Meteorological Society, \url{2023-03-29}.\\{}\\{}
Knapp KR, Diamond HJ, J.P. K, Kruk MC, Schreck CJ (2018).
``International Best Track Archive for Climate Stewardship (IBTrACS) Project, Version 4.''
\emph{NOAA National Centers for Environmental Information}.
\Rhref{https://doi.org/10.1175/2009BAMS2755.1}{doi:10.1175\slash{}2009BAMS2755.1}.
\end{References}
\inputencoding{utf8}
\HeaderA{d4z4}{Autosomal macrosatellite repeats d4z4}{d4z4}
\keyword{datasets}{d4z4}
%
\begin{Description}\relax
Macrosatellite repeats D4Z4 in the subtelomere of chromosome 4q.\\{}
Locus (hg18): 4q35.2 \\{}
Unit (kb): 3.3 \\{}
Restriction enzyme: EcoRI + HindIII/EcoRI + BlnI/XapI \\{}
Encoded product : DUX4
\end{Description}
%
\begin{Usage}
\begin{verbatim}
d4z4
\end{verbatim}
\end{Usage}
%
\begin{Format}
A vector of counts with 410 elements.
\end{Format}
%
\begin{References}\relax
Schaap M, Lemmers RJ, Maassen R, van der Vliet PJ, Hoogerheide LF, van Dijk HK, Basturk N, de Knijff P, van der Maarel SM (2013).
``Genome-wide analysis of macrosatellite repeat copy number variation in worldwide populations: evidence for differences and commonalities in size distributions and size restrictions.''
\emph{BMC Genomics}, \bold{14}(1), 143.
ISSN 1471-2164, \Rhref{https://doi.org/10.1186/1471-2164-14-143}{doi:10.1186\slash{}1471\-2164\-14\-143}.
\end{References}
\inputencoding{utf8}
\HeaderA{discrete\_MF}{Mode-finding algorithm for mixture of discrete distributions}{discrete.Rul.MF}
%
\begin{Description}\relax
Mode-finding algorithm for mixture of discrete distributions
\end{Description}
%
\begin{Usage}
\begin{verbatim}
discrete_MF(
  mcmc,
  data,
  pars_names,
  dist = "NA",
  pmf_func = NULL,
  type = "all",
  show_plot = FALSE
)
\end{verbatim}
\end{Usage}
%
\begin{Arguments}
\begin{ldescription}
\item[\code{mcmc}] Vector of estimated mixture parameters

\item[\code{data}] Vector of observations used for estimating the mixture

\item[\code{pars\_names}] Names of the mixture parameters; first element should 
correspond to the mixture proportions.

\item[\code{dist}] String indicating the distribution of the mixture components
Currently supports "poisson" and "shifted\_poisson"; default is "NA"; only
use this argument if you have used Poisson and shifted Poisson distributions
identical to the one used in the package.

\item[\code{pmf\_func}] Pmf of the mixture components associated with the mcmc draws
(if mcmc estimation has not been carried out with BayesMultiMode); default is null

\item[\code{type}] Type of modes, either unique or all (the latter includes flat modes); default is "all"

\item[\code{show\_plot}] If true show the data and estimated modes; default is false
\end{ldescription}
\end{Arguments}
%
\begin{Value}
Vector of estimated modes
\end{Value}
%
\begin{References}\relax
Schaap M, Lemmers RJ, Maassen R, van der Vliet PJ, Hoogerheide LF, van Dijk HK, Basturk N, de Knijff P, van der Maarel SM (2013).
``Genome-wide analysis of macrosatellite repeat copy number variation in worldwide populations: evidence for differences and commonalities in size distributions and size restrictions.''
\emph{BMC Genomics}, \bold{14}(1), 143.
ISSN 1471-2164, \Rhref{https://doi.org/10.1186/1471-2164-14-143}{doi:10.1186\slash{}1471\-2164\-14\-143}.
\end{References}
%
\begin{Examples}
\begin{ExampleCode}
# Example with the poisson distribution ====================================
lambda = c(0.1,10)
p = c(0.5,0.5)
params = c(eta = p, lambda = lambda)
pars_names = c("eta", "lambda")
dist = "poisson"

data = c(rpois(p[1]*1e3, lambda[1]),
         rpois(p[2]*1e3, lambda[2]))

modes = discrete_MF(params, data = data, pars_names = pars_names, dist = dist)

# Example with an arbitrary distribution ===================================
mu = c(20,5)
size = c(20,0.5)
p = c(0.5,0.5)
params = c(eta = p, mu = mu, size = size)
pars_names = c("eta", "mu", "size")

data = c(rnbinom(p[1]*1e3, mu = mu[1], size = size[1]),
         rnbinom(p[2]*1e3, mu = mu[2], size = size[2]))

pmf_func <- function(x, pars) {
  dnbinom(x, mu = pars["mu"], size = pars["size"])
}

modes = discrete_MF(params, data = data, pars_names = pars_names, pmf_func = pmf_func)

\end{ExampleCode}
\end{Examples}
\inputencoding{utf8}
\HeaderA{fixed\_point}{Modal fixed-point algorithm}{fixed.Rul.point}
%
\begin{Description}\relax
Algorithm for estimating modes in mixture of Normal distributions.
\end{Description}
%
\begin{Usage}
\begin{verbatim}
fixed_point(mcmc, data, pars_names, tol_x = sd(data)/10, show_plot = F)
\end{verbatim}
\end{Usage}
%
\begin{Arguments}
\begin{ldescription}
\item[\code{mcmc}] Vector of estimated mixture parameters

\item[\code{data}] Vector of observations used for estimating the mixture

\item[\code{pars\_names}] Names of the mixture parameters; first element should 
correspond to the mixture proportions; second to the mean; third to the 
standard deviation.

\item[\code{tol\_x}] Tolerance parameter for distance in-between modes; default is sd(data)/10; if two modes are closer than tol\_x, only the first estimated mode is kept.

\item[\code{show\_plot}] If true show the data and estimated modes; default is false
\end{ldescription}
\end{Arguments}
%
\begin{Value}
Vector of estimated modes
\end{Value}
%
\begin{References}\relax
Carreira-Perpinan MA (2000).
``Mode-finding for mixtures of Gaussian distributions.''
\emph{IEEE Transactions on Pattern Analysis and Machine Intelligence}, \bold{22}(11), 1318--1323.
ISSN 1939-3539, \Rhref{https://doi.org/10.1109/34.888716}{doi:10.1109\slash{}34.888716}, Conference Name: IEEE Transactions on Pattern Analysis and Machine Intelligence.
\end{References}
%
\begin{Examples}
\begin{ExampleCode}
mu = c(0,5)
sigma = c(1,2)
p = c(0.5,0.5)

data = c(rnorm(p[1]*100, mu[1], sigma[1]), rnorm(p[2]*100, mu[2], sigma[2]))
params = c(eta = p, mu = mu, sigma = sigma)
pars_names = c("eta", "mu", "sigma")
modes = fixed_point(params, data, pars_names)

\end{ExampleCode}
\end{Examples}
\inputencoding{utf8}
\HeaderA{galaxy}{Galaxy series}{galaxy}
\keyword{datasets}{galaxy}
%
\begin{Description}\relax
Velocity at which 82 galaxies in the Corona Borealis region are moving away from our galaxy, scaled by 1000.
\end{Description}
%
\begin{Usage}
\begin{verbatim}
galaxy
\end{verbatim}
\end{Usage}
%
\begin{Format}
An object of class \code{numeric} of length 82.
\end{Format}
%
\begin{Source}\relax
https://people.maths.bris.ac.uk/\textasciitilde{}mapjg/mixdata
\end{Source}
%
\begin{References}\relax
Richardson S, Green PJ (1997).
``On Bayesian Analysis of Mixtures with an Unknown Number of Components.''
\emph{Journal of the Royal Statistical Society. Series B (Methodological)}, \bold{59}(4), pp. 731--792.
ISSN 00359246.
\end{References}
\inputencoding{utf8}
\HeaderA{MEM}{Modal EM algorithm (MEM)}{MEM}
%
\begin{Description}\relax
Algorithm to find modes in mixture of continuous distributions.
\end{Description}
%
\begin{Usage}
\begin{verbatim}
MEM(
  mcmc,
  data,
  pars_names,
  dist = "NA",
  pdf_func = NULL,
  tol_x = sd(data)/10,
  show_plot = FALSE
)
\end{verbatim}
\end{Usage}
%
\begin{Arguments}
\begin{ldescription}
\item[\code{mcmc}] Vector of estimated mixture parameters

\item[\code{data}] Vector of observations used for estimating the mixture

\item[\code{pars\_names}] Names of the mixture parameters; the first element of 
this vector should be the name of the mixture proportions. If you have used 
the skew normal of Azzalini, then the second element should correspond to the location,
the third to the scale and the fourth to the shape.

\item[\code{dist}] String indicating the distribution of the mixture components; default is "NA"
Currently supports "normal" and "skew\_normal"; not needed if pdf\_func is provided

\item[\code{pdf\_func}] Pdf of the mixture components associated with the mcmc draws
(if mcmc estimation has not been carried out with BayesMultiMode); default is null

\item[\code{tol\_x}] Tolerance parameter for distance in-between modes; default is sd(data)/10; if two modes are closer than tol\_x, only the first estimated mode is kept.

\item[\code{show\_plot}] If true show the data and estimated modes; default is false
\end{ldescription}
\end{Arguments}
%
\begin{Value}
Vector of estimated modes
\end{Value}
%
\begin{References}\relax
Li J, Ray S, Lindsay BG (2007).
``A Nonparametric Statistical Approach to Clustering via Mode Identification.''
\emph{Journal of Machine Learning Research}, \bold{8}, 1687-1723.\\{}\\{}
Azzalini A (1985).
``A Class of Distributions Which Includes the Normal Ones.''
\emph{Scandinavian Journal of Statistics}, \bold{12}(2), 171--178.
ISSN 0303-6898, Publisher: [Board of the Foundation of the Scandinavian Journal of Statistics, Wiley], \url{2023-03-13}.
\end{References}
%
\begin{Examples}
\begin{ExampleCode}

# Example with the skew normal =============================================
xi = c(0,6)
omega = c(1,2)
alpha = c(0,0)
p = c(0.8,0.2)
params = c(eta = p, xi = xi, omega = omega, alpha = alpha)
pars_names = c("eta", "xi", "omega", "alpha")
dist = "skew_normal"

data = c(sn::rsn(p[1]*100, xi[1], omega[1], alpha[1]),
         sn::rsn(p[2]*100, xi[2], omega[2], alpha[2]))

modes = MEM(params, data = data, pars_names = pars_names, dist = dist)

# Example with an arbitrary distribution ===================================
xi = c(0,6)
omega = c(1,2)
alpha = c(0,0)
nu = c(3,100)
p = c(0.8,0.2)
params = c(eta = p, mu = xi, sigma = omega, xi = alpha, nu = nu)
pars_names = c("eta", "mu", "sigma", "xi", "nu")

pdf_func <- function(x, pars) {
  sn::dst(x, pars["mu"], pars["sigma"], pars["xi"], pars["nu"])
}

data = c(sn::rst(p[1]*100, xi[1], omega[1], alpha = alpha[1], nu = nu[1]),
         sn::rst(p[2]*100, xi[2], omega[2], alpha = alpha[2], nu = nu[2]))

modes = MEM(params, pars_names = pars_names, data = data, pdf_func = pdf_func)

\end{ExampleCode}
\end{Examples}
\inputencoding{utf8}
\HeaderA{new\_BayesMixture}{Creating a S3 object of class `BayesMixture`}{new.Rul.BayesMixture}
%
\begin{Description}\relax
Function for creating an object of class `bayes\_estimation()` which can subsequently be used as argument in `bayes\_mode()`.
\end{Description}
%
\begin{Usage}
\begin{verbatim}
new_BayesMixture(
  mcmc,
  data,
  K,
  burnin,
  dist = "NA",
  pars_names,
  pdf_func = NULL,
  dist_type
)
\end{verbatim}
\end{Usage}
%
\begin{Arguments}
\begin{ldescription}
\item[\code{mcmc}] A matrix of MCMC draws

\item[\code{data}] A vector containing the data used for estimating the model and generating the MCMC draws

\item[\code{K}] Number of mixture components

\item[\code{burnin}] Number of draws to discard as burnin

\item[\code{dist}] Distribution family of the mixture components supported by
the package (e.g. "normal", "student", "skew\_normal" or "shifted\_poisson")

\item[\code{pars\_names}] Names of the mixture parameters; first element should 
correspond to the mixture proportions.

\item[\code{pdf\_func}] Pdf or pmf of the mixture components;
this input is used only if dist\_name is invalid or NULL.

\item[\code{dist\_type}] Either "continuous" or "discrete"
\end{ldescription}
\end{Arguments}
%
\begin{Value}
A list of class `BayesMixture` containing :
\begin{itemize}

\item{} data - Same as argument
\item{} dist\_type - Same as argument
\item{} pars\_names - Same as argument
\item{} mcmc - Matrix of MCMC draws where the rows corresponding to burnin have been discarded
\item{} mcmc\_all - Original matrix of MCMC draws

\end{itemize}

\end{Value}
\inputencoding{utf8}
\HeaderA{plot.BayesMixture}{Plot an estimated mixture for a given number of draws with an histogram of the data.}{plot.BayesMixture}
%
\begin{Description}\relax
Plot an estimated mixture for a given number of draws with an histogram of the data.
\end{Description}
%
\begin{Usage}
\begin{verbatim}
## S3 method for class 'BayesMixture'
plot(x, max_size = 250, transparency = 0.1, ...)
\end{verbatim}
\end{Usage}
%
\begin{Arguments}
\begin{ldescription}
\item[\code{x}] An object of class BayesMixture.

\item[\code{max\_size}] The number of MCMC draws to plot.

\item[\code{transparency}] transparency of the density lines. Default is 0.1. Should be greater than 0 and below or equal to 1.

\item[\code{...}] Not used.
\end{ldescription}
\end{Arguments}
\inputencoding{utf8}
\HeaderA{plot.BayesMode}{Plot Bayesian mode estimates.}{plot.BayesMode}
%
\begin{Description}\relax
Plot Bayesian mode estimates.
\end{Description}
%
\begin{Usage}
\begin{verbatim}
## S3 method for class 'BayesMode'
plot(x, graphs = c("p1", "number", "loc"), ...)
\end{verbatim}
\end{Usage}
%
\begin{Arguments}
\begin{ldescription}
\item[\code{x}] An object of class BayesMode.

\item[\code{graphs}] which plot to show ? Default is all three c("p1", "number", "loc").

\item[\code{...}] Not used.
\end{ldescription}
\end{Arguments}
\inputencoding{utf8}
\HeaderA{summary.BayesMode}{Summary of Bayesian mode estimates.}{summary.BayesMode}
%
\begin{Description}\relax
Summary of Bayesian mode estimates.
\end{Description}
%
\begin{Usage}
\begin{verbatim}
## S3 method for class 'BayesMode'
summary(object, ...)
\end{verbatim}
\end{Usage}
%
\begin{Arguments}
\begin{ldescription}
\item[\code{object}] An object of class BayesMode.

\item[\code{...}] Not used.
\end{ldescription}
\end{Arguments}
\printindex{}
\end{document}
